\section{Considerações Finais}
Este estudo demonstrou que a aplicação de aprendizado de máquina na análise de cardiotocografia (CTG) é uma estratégia eficaz para auxiliar no diagnóstico precoce de riscos à saúde fetal. Após comparar cinco algoritmos distintos, a Árvore de Decisão consolidou-se como o modelo mais competente para esta tarefa, alcançando um F1-Score de 93.4\% e superando abordagens como Redes Neurais e Naive Bayes.

A superioridade da Árvore de Decisão neste cenário não foi acidental. A análise dos dados revelou que exames de CTG possuem distribuições complexas e eventos clínicos extremos (como desacelerações abruptas) que atuam como \textit{outliers}. Enquanto modelos baseados em estatística pura ou distância tiveram dificuldade com essas características, a estrutura hierárquica da Árvore conseguiu isolar e interpretar esses padrões críticos com alta precisão. Além da performance, este modelo oferece a vantagem crucial da interpretabilidade, permitindo que a equipe médica visualize as regras lógicas de decisão, o que aumenta a confiança no diagnóstico assistido por computador.

Em contrapartida, modelos que assumem a independência entre variáveis, como o Naive Bayes, apresentaram desempenho inferior, confirmando a existência de correlações complexas entre os parâmetros fisiológicos do feto que não podem ser ignoradas.

Para trabalhos futuros, os resultados aqui obtidos sugerem um caminho promissor para a integração destes algoritmos em sistemas de monitoramento hospitalar em tempo real, atuando como alertas automáticos para a equipe de obstetrícia. Além disso, investigações futuras podem focar na otimização de modelos para dispositivos portáteis e de baixo custo, democratizando o acesso a diagnósticos de alta precisão em regiões com escassez de especialistas.
